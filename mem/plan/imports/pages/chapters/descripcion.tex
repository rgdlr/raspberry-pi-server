\section{Descripción} \label{chapters:descripcion:sections:descripcion}

El proyecto presentado tiene la finalidad principal de crear un \textbf{servidor doméstico sobre el hardware de una Raspberry Pi}, teniendo en cuenta varias características fundamentales en su presentación final: instalación automatizada, facilidad de configuración y alojamiento, y prestación de servicios. Cada característica posee distintos motivos de implementación a nivel de usuario:

\begin{enumerate}
    \item \textbf{Automatización}: el proyecto podrá ser fácilmente replicable a través de una instalación simple, ofreciendo el código fuente generado junto a su documentación a través de un repositorio. De este modo, se ofrece a un usuario básico la posibilidad de utilizarlo sin conocimientos previos, además del conocimiento necesario para poder personalizarlo.
    \item \textbf{Configuración}: el medio de configuración inicial será a través de una interfaz web simple y moderna, que ejecutará sobre el propio servidor tras su instalación. Además, se podrán añadir otros métodos de configuración posteriormente, como podría ser a través de un bot de Telegram, comandos de voz, etc. Esto, permitirá a un usuario intermedio poder adaptar las configuraciones a sus necesidades.
    \item \textbf{Servicios}: los servicios ofrecidos inicialmente serán los de alojamiento web y base de datos. Además, se podrán añadir otros servicios como VPN, FTP, NFS, etc. llegando a formar un producto similar a un NAS de alto presupuesto, de manera doméstica y fácilmente configurable. A través de esta parte, un usuario avanzado podrá implementar sus propias características al servidor.
\end{enumerate}

Además, se implementará una funcionalidad que permitirá a un desarrollador web, en fase de desarrollo, \textbf{visualizar su proyecto desde distintos dispositivos en la red} (equipo de torre o portátil, tablet, smartphone, televisión, etc.), creando un entorno de desarrollo altamente productivo a la hora de comprobar el \textit{responsive design} o diseño web adaptable para los distintos tipos de pantalla.

Teniendo en cuenta esta parte del proyecto como la más esencial y costosa, se plantea la posibilidad de \textbf{ampliarlo} con varias tareas extra, entre las cuales se encuentra una de alta utilidad para la universidad: generar, con el hardware utilizado en el proyecto, un producto autoconfigurado para \textbf{acceder a los escritorios virtuales} de la facultad.

Este es un proyecto que generará un \textbf{producto útil para un desarrollador}, con alta relación calidad/precio, energéticamente eficiente y ampliable en utilidades.

\clearpage
