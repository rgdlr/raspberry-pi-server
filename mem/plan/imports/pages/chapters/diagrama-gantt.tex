\section{Diagrama de Gantt} \label{chapters:diagrama-gantt:sections:diagrama-gantt}

Teniendo en cuenta las \textbf{tareas} descritas, el Diagrama de Gantt asociado sería el siguiente:

\begin{figure}[ht]
    \begin{ganttchart}[
        include title in canvas=false,
        expand chart=\textwidth,
        hgrid={*1{gray, dashed}},
        vgrid={*1{gray, dashed}},
        title/.style={draw=none, fill=none},
        title/.append style={opacity=.8},
        title height=1,
        y unit title=.7cm,
        group left shift=0,
        group right shift=0,
        group height=.0125cm,
        group peaks height=.005cm,
        group label font=\bfseries\small,
        group progress label font=\scriptsize\color{black!35},
        group/.append style={draw=black, fill=black!60},
        group incomplete/.append style={draw=black, fill=black!15},
        bar height=.015cm,
        bar label font=\scriptsize,
        bar progress label font=\scriptsize\color{black!35},
        bar/.append style={draw=black, fill=black!30},
        bar incomplete/.append style={fill=black!7},
        milestone label font=\bfseries\small,
        today label=Hoy,
        today label font=\bfseries\scriptsize,
        today=3,
        progress=today,
        progress label text={$\pgfmathprintnumber[precision=0, verbatim]{#1}\%\hspace{.1cm}completado$},
    ]{1}{17}
        \gantttitle[title label font=\color{black!60}]{Semana (desde el 14 de febrero)}{17} \ganttnewline
        \gantttitlelist[y unit title=.7cm]{1,...,17}{1} \ganttnewline

        \ganttgroup[progress=today]{Plan Trabajo}{1}{3}[name=plan-trabajo] \ganttnewline
        \ganttbar[progress=today]{Documentación}{1}{3}[name=plan-trabajo-documentacion] \ganttnewline
        
        \ganttgroup[progress=today]{Desarrollo}{3}{14}[name=desarrollo] \ganttnewline
        \ganttbar[progress=today]{Análisis}{3}{6}[name=desarrollo-análisis] \ganttnewline
        \ganttbar[progress=today]{Diseño}{3}{7}[name=desarrollo-diseño] \ganttnewline
        \ganttbar[progress=today]{Implementación}{5}{14}[name=desarrollo-implementacion] \ganttnewline
        \ganttbar[progress=today]{Pruebas}{5}{14}[name=pruebas] \ganttnewline

        \ganttgroup[progress=today]{Memoria}{4}{15}[name=memoria] \ganttnewline
        \ganttbar[progress=today]{Seguimiento}{4}{10}[name=memoria-seguimiento] \ganttnewline
        \ganttbar[progress=today]{Final}{7}{15}[name=memoria-final] \ganttnewline
        
        \ganttgroup[progress=today]{Presentación}{16}{17}[name=presentacion] \ganttnewline
        \ganttbar[progress=today]{Diapositivas}{16}{17}[name=diapositivas] \ganttnewline
        
        \ganttmilestone[progress label text=\hspace{.2cm}PT]{Entregas}{3}
        \ganttmilestone[progress label text=\hspace{.2cm}MS]{}{10}
        \ganttmilestone[progress label text=\hspace{.2cm}MF]{}{15}
        \ganttmilestone[progress label text=\hspace{.2cm}PR]{}{17}
    \end{ganttchart}
    \caption{Diagrama de Gantt}
    \label{chapters:gantt:sections:diagrama-gantt:figure:diagrama-gantt}
\end{figure}

Al final de las fases de \textbf{Análisis} y \textbf{Diseño}, se realizará un \textbf{desglose} de las tareas de manera más exhaustiva, ya que, es en ese momento, cuando se habrá decidido cómo llevar a cabo el \textbf{desarrollo} del proyecto con una mayor exactitud.

\clearpage
