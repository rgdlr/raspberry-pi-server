\section{Resumen} \label{chapters:resumen:sections:resumen}

El trabajo realizado hasta la fecha de entrega de la actual \textbf{Memoria de Seguimiento} (\date{24 de abril de 2022}), se corresponde con las primeras funcionalidades del proyecto y puede verse reflejado en las figuras de \hyperref[chapters:revision:sections:revision:subsections:diagrama-gantt:figure:plan]{Plan}, \hyperref[chapters:revision:sections:revision:subsections:diagrama-gantt:figure:desarrollo]{Desarrollo} y \hyperref[chapters:revision:sections:revision:subsections:diagrama-gantt:figure:memoria]{Memoria} asociadas a los diagramas de Gantt desarrollados.

\textbf{A pesar de los cambios} en los objetivos del proyecto que han surgido a lo largo de las primeras semanas de análisis e investigación, cabe destacar que, actualmente, el trabajo avanza \textbf{según lo previsto}.

Las \textbf{estimaciones iniciales} parecen \textbf{acertadas} y, teniendo en cuenta los avances actuales, el proyecto concluirá dentro de las fechas marcadas.

Como se comenta al inicio de este apartado, los avances pueden observarse en los diagramas de Gantt de la sección \ref{chapters:revision:sections:revision:subsections:diagrama-gantt}, y las funcionalidades \textbf{completadas} son:

\begin{enumerate}[itemsep=0em]
    \item Instalación del sistema operativo.
    \item Configuración del sistema operativo.
    \item Configuración de acceso al servidor desde la red local a través de SSH.
    \item Configuración de acceso al servidor desde internet a través de SSH.
\end{enumerate}

Por otro lado, las funcionalidades que están actualmente \textbf{en desarrollado} de forma paralela son:

\begin{enumerate}[itemsep=0em]
    \item Documentación de uso.
    \item Securización del servidor.
    \item Servicios de alojamiento web y base de datos.
    \item Despliegue de aplicaciones web de forma remota.
    \item Script interactivo de instalación y configuración del proyecto.
    \item Script automatizado de instalación y configuración del proyecto.
\end{enumerate}


Finalmente, una funcionalidad (secundaria) aún \textbf{por hacer} para la entrega final es:

\begin{enumerate}[itemsep=0em]
    \item Integración de escritorio virtual de la UPM en Raspberry Pi.
\end{enumerate}


\clearpage
